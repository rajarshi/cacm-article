\documentclass{sig-alternate}

\begin{document}

\title{Supplement History of Chem(o)informaqtics as part of Cheminformatics: The Computer Science of Chemical Discovery}
\numberofauthors{9}
\author{
\alignauthor
Joerg Kurt Wegner\\
       \affaddr{Tibotec BVBA}\\
       \affaddr{Turnhoutseweg 30}\\
       \affaddr{2340 Beerse Turnhout, Belgium}\\
       \email{jwegner@its.jnj.com}
% 2nd. author
\alignauthor
Aaron Sterling\\
       \affaddr{Department of Computer Science}\\
       \affaddr{Iowa State University}\\
       \affaddr{Ames, Iowa, USA}\\
       \email{sterling@iastate.edu}
% 3rd author
\alignauthor
Rajarshi Guha\\
\affaddr{NIH Center for Translational Therapeutics}\\
\affaddr{9800 Medical Center Drive}\\
\affaddr{Rockville, MD 20850}\\
\email{guhar@mail.nih.gov}
}

\additionalauthors{Additional authors:
Andreas Bender (University of Cambridge, email: {\texttt{andreas.bender@cantab.net}}),
Jean-Loup Faulon (University of Evry, email: {\texttt{Jean-Loup.Faulon@issb.genopole.fr}}),
Janna Hastings (European Bioinformatics Institute, Cambridge, UK, email: {\texttt{janna.hastings@gmail.com}}),
Noel O'Boyle (University College Cork, Cork, Ireland, email: {\texttt{baoilleach@gmail.com}}),
John Overington (European Bioinformatics Institute, Cambridge, UK, email: {\texttt{jpo@ebi.ac.uk}}),
Herman Van Vlijmen (Tibotec, Beerse, Belgium, email: {\texttt{hvvlijme@its.jnj.com}}), and
Egon Willighagen (Karolinska Institutet, Stockholm, Sweden, email: {\texttt{egon.willighagen@ki.se}})
.}
\date{25 June 2011}
\maketitle

\section{History of Chem(o)informatics}
The aim of this brief review of the history of cheminformatics is to put the
content of the main article into a broader perspective. Therefore it
might be useful to provide some definitions of cheminformatics, which
might also shed some light on the founding principles of the
field. Chen \cite{Chen2006} and Brown \cite{brown2009} cited various
definitions of pioniers in the field and Brown concluded rightfully:
'\textit{The differences in definitions (of the term cheminformatics)
  are largely a result of the types of analyses that particular
  scientists practice and no single definition is intended to be
  all-encompassing}.'
\begin{itemize}
\item '\textit{The mixing of information resources to transform data
    into information, and information into knowledge, for the intended
    purpose of making better decisions faster in the arena of drug
    lead identification and optimization.}' Frank K. Brown, 1998.
\item '[Chemoinformatics involves]... \textit{the computer
    manipulation of two- or threedimensional chemical structures and
    excludes textual information. This distinguishes the term from
    chemical information, largely a discipline of chemical librarians
    and does not include the development of computational methods.}'
  Peter Willett, 2002.
\item '\textit{...the application of informatics to solve chemical
    problems.}' and '\textit{chemoinformatics makes the point that
    you�re using one scientific discipline to understand another
    scientific discipline.}' Johann Gasteiger, 2002
\end{itemize}
Furthermore is it useful to clarify some other terms and how
cheminformatics might relate to them:
\begin{itemize}
\item \textbf{Quantum chemistry} is something most chemists would
  relate with 'theoretical chemistry'. In short, it allows to describe
  chemical molecules on an electron density level, which allows also
  to describe physical properties, e.g infrared spectra of a chemical
  compound.  The results of quantum chemistry calculation are used in
  molecular modeling (energy force field parameters) and
  cheminformatics (atom partial charges, or full molecule properties,
  e.g molecule polarizabilities).
\item \textbf{Molecular modeling} or \textbf{Computational chemistry} considers typically energy calculations of single molecules or interactions
with other molecules, e.g. proteins. 'Molecular docking' belongs into this category, since it places a chemical molecule in the active site of enzymes
and calculates energy/interaction scores between a chemical molecule and the protein. Surely are there various cheminformatics concepts required to
perform such a task. Still, strictly speaking is cheminformatics rather on the chemical molecule side than on the protein side, though are both sides
affecting each other. This area is typically strongly related to structural biology and technologies like XRay crystallography and 
NMR spectra structure determination. 
\item \textbf{Chemometrics} can be considered as applied cheminformatics to the field of chemical/physical analysis, e.g. infrared, mass and NMR spectra analysis.
\item '\textbf{Translational research/medicine} aims to improve the health and longevity of the world's populations and depends on developing broad-based 
teams of scientists and scholars who are able to focus their efforts to link basic scientific discoveries with the arena of clinical investigation, 
and translating the results of clinical trials into changes in clinical practice, informed by evidence from the social and political sciences.' (Source:Wikipedia).
Since clinical trials can be connected to drugs, chemical molecules, even all of translational research and biobanking could be considered as a field
of cheminformatics, which would be on the other hand just a small contribution to a full knowledge management in this field \cite{Szalma2010}.
\end{itemize}
If we consider all information, analysis, and optimization of a chemical molecule as 'cheminformatics', then the field is very large, since a chemical
molecule plays the central role in many related disciplines. Still, for keeping it practical the aim of this review is not being exhaustive, but rather to 
point the reader to more detailed references and to highlight some historic milestones. 

One of the most comprehensive cheminformatic series was published by Prof. Gasteiger in 2003 \cite{Gasteiger2003}, one of the pioniers in the field.
Compacter introductory books followed from Gasteiger/Engel \cite{gasteigerengel2003} and Leach/Gillet \cite{leachgillet2007}.
The books of Bajorath \cite{Bajorath2004} and Oprea \cite{oprea2005} focus on the applications of cheminformatics in drug design.
The first books discussing chemical graph theory and algorithms were published in 1989 by Zupan \cite{zupan1989}, 
1991 by Bonchev/Rouvrey \cite{bonchevrouvrey1991,bonchevrouvrey2003} and 1992 by Trinastic \cite{Trinajstic1992}. 
The most recent one was published by Faulon/Bender \cite{faulon2010}.
Other books focus more on specific fields/applications like mathematical challenges \cite{mathchallenges1995}, 
molecular diversity \cite{moleculardiversity1999}, factor analysis \cite{Malinowski2002}, evolutionary algorithms \cite{clark2000},
molecular descriptors \cite{todeschini2000}.

\section{Historic milestones}
The first chemical graphs were drawn by the Scottish chemist Willam Cullen in 1758 \cite{bonchevrouvrey1991}, which called them initially affinity graphs.
After developing the concept of bonds between atoms (Couper, 1858) the first chemical graphs occured in publications 
of Brown in 1864 and Cayley in 1874 \cite{bonchevrouvrey1991,brown2009}. 

The year 1946 may be regarded as the birth year of chemoinformatics \cite{Chen2006}: '\textit{In 1946 King et al.\cite{kct1946} published an article illustrating 
the use of IBM's business accounting machines in carrying out the construction of the rotational spectra of asymmetric rotors
by the evaluation of mathematical equations for line position and line intensity.}'

The first records of managing chemical information go back to chemical literature being indexed since the year 1771.
In 1881 the first edition of \textit{Beilstein's Handbuch der Organischen Chemie} ecyclopedia was 
published \cite{polanski2009} registering 1500 chemical compounds. The short term 'chemoinformatics' was coined by Brown 
in 1998\cite{brown1998}, while the \textit{Journal of Chemical Documentation} (founded in 1961) changed already its name in 1975 to 
\textit{Journal of Chemical Information and Computer Sciences} to reflect the tight connection between chemical information 
and computer science. The Chemical Abstract Service (CAS) of the American Chemical Society (ACS) formed in 1955 a research department
and starting from 1965 they provided their Chemical Registry System, and in 1968 they made the first computer-readable file of all
abstracted documents available \cite{Chen2006}.

Ray and Kirsch published in 1957 the first substructure searching algorithm for allowing to find chemical records via computers \cite{RayKirsch1957}.

One of the key algorithms for molecular graph canonicalization is the Morgan algorithm \cite{Morgan1965}. It will search for the most central bond
in a graph and allows then to traverse other atoms, resolve ties, and assigning new atom indices. Still, there are various options how different 
software implementations are handling the 'ties' for assigning new atom indices. So, by using the canonicalization of multiple software implementations and chemical
expert systems can be even used to create randomized (canonical, only when sticking to one software) molecular graphs, which can be used as stochastic input for 
other algorithms \cite{cokfl06}. The central bond is also being used to create 3D conformational search trees \cite{Schwab2001}. 

In 1965 the DENDRAL expert system started with the aim to automatically determine the structure of an unknown chemical compound from the 
corresponding mass spectrum \cite{Gray1986}. The system is also often being cited as one of the earliest artificial intelligence and expert systems  \cite{Chen2006}.

The first system supporting chemical synthesis planning was OCSS (Organic Chemical Simulation of Syntheses) developed by Corey and Wipke in 1969 \cite{CoreyWipke1969}.

In the early 1970s the collaboration between Lempel-Ziv-Welch has led to the LZW algorithm, since the Cheminformatician Lynch was
discussing chemical fragment screening approaches with the computer scientists Lempel and Ziv \cite{brown2009}.

In 1978 Gasteiger/Marsili published a fast algorithm to calculate partial charges in organic molecules by a Partial Equalization of Orbital 
Electronegativities (PEOE) \cite{gm78}, which is till today one of the gold-standards to calculate partial charges of atoms.

Instead of converting molecular graphs to a vector representation for applying machine learning methods, it is also possible to use direct 
molecular graph mining methods \cite{okada2006}. Some of the first published methods use neural networks \cite{kireev1995}, 
inductive logic programming \cite{yh02a}, or graph kernels \cite{kti03}.

Molecular docking of chemical molecules into proteins is an important application in molecular modeling \cite{Leach2001}. It is noteworthy that in some cases
molecular shape matching methods alone (cheminformatics), can outperform energy-based molecular docking algorithms (molecular modeling) \cite{hsn07}.

\bibliographystyle{abbrv}
\bibliography{paper}

\end{document}
