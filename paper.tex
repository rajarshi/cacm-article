\documentclass{sig-alternate}

\begin{document}

\title{Chemoinformatics: The Computer Science of Chemistry}
\numberofauthors{3}
\author{
\alignauthor
Ben Trovato\titlenote{Dr.~Trovato insisted his name be first.}\\
       \affaddr{Institute for Clarity in Documentation}\\
       \affaddr{1932 Wallamaloo Lane}\\
       \affaddr{Wallamaloo, New Zealand}\\
       \email{trovato@corporation.com}
% 2nd. author
\alignauthor
G.K.M. Tobin\titlenote{The secretary disavows
any knowledge of this author's actions.}\\
       \affaddr{Institute for Clarity in Documentation}\\
       \affaddr{P.O. Box 1212}\\
       \affaddr{Dublin, Ohio 43017-6221}\\
       \email{webmaster@marysville-ohio.com}
% 3rd author
\alignauthor
Rajarshi Guha\\
\affaddr{NIH Center for Translational Therapeutics}\\
\affaddr{9800 Medical Center Drive}\\
\affaddr{Rockvile, MD 20850}\\
\email{guhar@mail.nih.gov}
}

\additionalauthors{Additional authors: John Smith (The Th{\o}rv{\"a}ld Group,
email: {\texttt{jsmith@affiliation.org}}) and Julius P.~Kumquat
(The Kumquat Consortium, email: {\texttt{jpkumquat@consortium.net}}).}
\date{30 July 1999}

\maketitle
\begin{abstract}
  One of the most prominent success stories in all of science over the last decade has
  been the advance of bioinformatics: interdisciplinary collaboration between
  computer scientists and molecular biologists led to the sequencing the human genome
  (among many other accomplishments), because researchers connected biological
  technologies to the theory of string algorithms. However, despite this great
  success, few computer scientists are familiar with a related (and older!)
  discipline -- chemoinformatics, the use of computers to discover new molecules with
  desired behavior, such as new pharmaceuticals or industrial solvents. Until
  recently, the data and techniques of chemoinformatics have been closely guarded
  secrets of companies whose financial success depended on being the first to produce
  the new "miracle molecule". Only within the last decade -- and largely because of
  chemists volunteering their time for an Open Science "movement" -- do researchers
  now have access to freely available software packages and databases of tens of
  millions of chemicals. As a result, chemists now confront hundreds of unsolved
  algorithmic problems that could not have been tackled a decade ago, but whose
  solutions are critical to research ranging from determining the behavior of small
  molecules in biological pathways, to finding a cure for malaria.
\end{abstract}

\category{H.4}{Information Systems Applications}{Miscellaneous}
\terms{Theory}
\keywords{cheminformatics, graphs, chemistry}

\section{Is Cheminformatics the New Bioinformatics?}


\bibliographystyle{abbrv}
\bibliography{paper}  

\end{document}